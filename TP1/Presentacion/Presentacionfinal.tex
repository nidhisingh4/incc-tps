\documentclass[10pt]{beamer}
%\documentclass[10pt,handout]{beamer}
\usepackage[spanish]{babel}
\usepackage[utf8x]{inputenc}
\usepackage{amsmath}
\usepackage{amsfonts}

%\usetheme{Frankfurt}
%\usetheme{Ilmenau}
%\usetheme{Luebeck}
\usetheme{Madrid}
%\usetheme{Warsaw}

\title[Identificaci\'on de Letras Utilizando Bubbles]{Detecci\'on de Rasgos en la Identificaci\'on de Letras Utilizando Bubbles}
\subtitle{Intr. a Neurociencia Cognitiva y Computacional}
\author[Miguel, Mail\'en, Christian]{Christian Cossio Mercado,\\Mail\'en G\'omez Mayol,\\Miguel Mart\'inez Soler}
\institute[FCEyN,UBA]{Departamento de Computaci\'on - FCEyN, UBA}
\date{31 de mayo de 2011}


\begin{document}

\begin{frame}%<handout:0>
\titlepage
\end{frame}

\section{Introducci\'on}
  \subsection{Objetivos}
      \begin{frame}
	\frametitle{Objetivo del experimento}
	\begin{itemize}
		\item \textbf{Identificar rasgos utilizados por las personas para identificar letras presentadas en distintas tipograf\'ias}\pause
		\item \alert{¿Cómo lo hacemos?}
	\end{itemize}
      \end{frame}

  \subsection{Todos Somos Sujetos}
      %estímulo máscara random
      \begin{frame}
	\frametitle{Todos Somos Sujetos}
	\begin{itemize}
		\item Vamos a intentar identificar algunas letras\ldots
	\end{itemize}
      \end{frame}

  %objetivos e hipótesis
\part[Antecedentes]{Revisi\'on de Antecedentes}
\frame{\partpage}

  \section{Antecedentes}
  %Pelli
	\begin{frame}
	\frametitle{Feature Detection and Letter Identification\\(Pelli et al., 2006)}
	\begin{columns}[t]
	\column{.60\textwidth}
	\begin{itemize}
		\item Conceptos de la identificación de letras y metodología experimental\pause
		\item Definición de complejidad (Attneave)
		  \begin{equation*}
		   \mathnormal{\text{complejidad}(l) = \frac{\text{per\'imetro}(l)^2}{\text{superficie}(l)}}
		  \end{equation*}\pause
		\item Relación eficiencia/complejidad
	\end{itemize}

	\column{.45\textwidth}

	\begin{figure}
	\includegraphics[width=\textwidth]{graficos/pelli4.png}
	\caption{Eficiencia vs complejidad para distintas tipografías}
	\end{figure}
	\end{columns}
  \end{frame}

%Gosselin
	\begin{frame}
	\frametitle{Bubbles: a technique to reveal the use of information in recognition task (Gosselin \& Schyns, 2001)}
	\begin{columns}[t]
	  \column{.65\textwidth}
	    \begin{itemize}
		\item Concepto de la técnica y del diseño del experimento
		\item<2->Generación de un estímulo
		    \begin{figure}
		      \includegraphics[width=0.9\textwidth]{graficos/estimuloGosseling.png}
		      \caption{Generaci\'on de un est\'imulo}
		    \end{figure}
	    \end{itemize}
	  \column{.35\textwidth}
	    \begin{itemize}
		  \item<2->Variables en juego
		  \only<2-2>{
		    \begin{itemize}
		      \item estímulo
		      \item dimensiones del estímulo
		      \item tamaño y cant. de burbujas
		      \item observadores
		    \end{itemize}
		  }
	    \end{itemize}
	    \uncover<3->{
		\begin{figure}
		  \includegraphics[width=\textwidth]{graficos/gosselin2.png}
		  \caption{Reconocimiento de expresión (ENEX) y género (GENDER)}
		\end{figure}
	    }
	  \end{columns}
	\end{frame}


	%Fiset
      \begin{frame}
	\frametitle{Features for Identification of Uppercase and Lowercase Letters (Fiset et al., 2008)}

	\begin{columns}[t]
	    \column{.4\textwidth}
	    \begin{itemize}
		\item Uso de Bubbles para identificación de letras
		\item<2-> 54 letras Arial
	    \end{itemize}	    
	  \uncover<3->{
	    \begin{figure}
		\includegraphics[width=.6\textwidth]{graficos/fiset1.png}
		\caption[Fiset et al]{Rasgos relevantes para humanos}
	    \end{figure}
	    
	    \column{.6\textwidth}
	    \begin{itemize}
		\item Humanos: Agregan 1 burbuja hasta llegar al 52\% de aciertos
		\item Obs.Ideal: Burbujas fijas, aumentan ruido hasta bajar al 52\% de aciertos
	    \end{itemize}	    
	    \begin{figure}
		\includegraphics[width=0.5\textheight]{graficos/fiset5.png}
		\caption{Importancia relativa de los rasgos}
	    \end{figure}
	  }
	\end{columns}
      \end{frame}

\part[Experimento]{Diseño del Experimento}
\frame{\partpage}
\section{Experimento}

  \subsection{Objetivo e Hip\'otesis}
      \begin{frame}
	\frametitle{Objetivo e Hip\'otesis}
	\begin{itemize}
		\item Identificar rasgos utilizados por las personas para identificar letras presentadas en distintas tipograf\'ias
      \end{itemize}	\pause
      \begin{block}{Hip\'otesis}
	  \begin{enumerate}
		\item\alert<3>{El uso de tipograf\'ias ampliamente conocidas facilita el reconocimiento de letras, aún cuando la persona no se da cuenta de ello}
		\item\alert<3>{La performance en el reconocimiento de las letras es inversamente proporcional a su complejidad}
		\item\alert<4>{Los rasgos de cada letra var\'ian de acuerdo a la tipograf\'ia que se est\'e utilizando}
		\item\alert<4>{Habrá cambios en los rasgos de la `n' por la incorporación de la `ñ'}
		\item\alert<4>{Se obtendrá rasgos similares a los encontrados en la bibliografía}
		\item\alert<4>{Un observador ideal utilizar\'a rasgos distintos a los que utiliza una persona para identificar letras}
	  \end{enumerate}
      \end{block}

      \end{frame}
  \subsection{Dise\~no}

	%Elección tipografías
	\begin{frame}
	\frametitle{Elecci\'on de tipograf\'ias}
	    \begin{figure}
		\includegraphics[width=\textwidth]{graficos/letras.png}\\ \pause
		\includegraphics[width=0.8\textheight]{graficos/complejidadesBoxplot.png}
		%\caption{Tipograf\'ias y sus Complejidades}
	    \end{figure}
	\end{frame}

	%RASGOS
	\begin{frame}
	\frametitle{Identificaci\'on de Rasgos}
	\begin{columns} [t]
	\column{.5\textwidth}
	\begin{figure}
	\includegraphics[width=.8\textwidth]{graficos/REFERENCIA.png}
	\caption{Identificaci\'on de rasgos para la letra \~n}
	\end{figure}
	\column{.5\textwidth}
	\begin{figure}
	\includegraphics[width=.5\textwidth]{graficos/letras.png}
	\caption{Uso relativo de los rasgos necesarios para identificar letras}
	\end{figure}
	\end{columns}
	\end{frame}

	%estímulos
	\begin{frame}
	\frametitle{Generaci\'on de Est\'imulos}
	    \begin{figure}
	    \includegraphics[width=\textwidth]{graficos/estimulofinal.png}
	    \caption{Armado del est\'imulo final}
	    \end{figure}
	\end{frame}

	%Jueves
	\begin{frame}
	\frametitle{Primer Diseño del Experimento: Jueves 12/5}
	    \begin{itemize}
		\item 13 sujetos (Gracias a todos, nuevamente!)
		\item Pocos bloques y ensayos (5 x 100, t $\approx$ 20min)
		\item Se completa una encuesta al terminar (performance, tipografías famosas)
		\item Muchas burbujas (todas las letras comienzan igual con la misma cantidad)
		\item Muy poca información {\bf:-(} (para la mayoría no se alcanza un valor cercano al 52\% de aciertos)\pause
		\item Muchos gastos en golosinas {\bf:-P}\pause
	    \end{itemize}
	\textbf{Posible Soluci\'on}: Ampliar la cantidad de ensayos y ajustar parámetros (bloques y burbujas)
	\end{frame}

	%Recauchutaje
	\begin{frame}
	\frametitle{Redise\~no del Experimento}
	    \begin{itemize}
		\item Más bloques por sujeto (17 x 100, t $\approx$ 1hr)
		\item Correcciones de errores menores (randoms, cantidad de burbujas, burbujas por banda)
		\item Mejora en la cantidad de burbujas inicial (mayor complejidad, mayor cantidad de burbujas iniciales)
		\item Filtrando casos en que no se llegó al 52\% \pause
		\item Se tiró los datos anteriores, reemplazando con los nuevos
	    \end{itemize}
	\end{frame}

	%Final
	\begin{frame}
	\frametitle{Datos Finales}
	    \begin{itemize}
		\item 6 sujetos
		\item Edades entre 21-33 años
		\item Con estudios universitarios
		\item 1700 ensayos por persona\pause
		\item Para completar datos \ldots\pause también fuimos sujetos (2500 ensayos)
	    \end{itemize}
	\end{frame}


% El uso de tipografías ampliamente conocidas facilita el reconocimiento de letras, aún cuando la persona no se da cuenta de ello
% La performance en el reconocimiento de las letras es inversamente proporcional a su complejidad
% Los rasgos de cada letra varían de acuerdo a la tipografía que se esté utilizando
% Habrá cambios en los rasgos de la `n' por la incorporación de la `ñ'
% Se obtendrá rasgos similares a los encontrados en la bibliografía
% Un observador ideal utilizará rasgos distintos a los que utiliza una persona para identificar letras

\part{Resultados}
  \frame{\partpage}
  \section{Gr\'aficos y Tablas}
 	\begin{frame}
	\frametitle{Letras famosas}

	\end{frame}
  
 	\begin{frame}
	\frametitle{Burbujas/Tiempo vs. Complejidad}

	\end{frame}

 	\begin{frame}
	\frametitle{Rasgos Detectados}

	\end{frame}

 	\begin{frame}
	\frametitle{Rasgos para \~N}

	\end{frame}


  \section{Conclusiones}
	\begin{frame}
	\frametitle{Conclusiones}
	    \begin{itemize}
	    \item Cantidad de respuestas necesarias (o estímulos a mostrar): 156.000= 3.9 días de experimentación continua.
	    \item Resulta una técnica útil para el muestreo de espacios sin limitación en la cantidad de dimensiones
	    \end{itemize}
	\end{frame}

  \section{Lecciones Aprendidas}
	\begin{frame}
	\frametitle{Lecciones Aprendidas}
	    \begin{itemize}
	    \item Cantidad de respuestas necesarias (o estímulos a mostrar): 156.000= 3.9 días de experimentación continua.
	    \item Resulta una técnica útil para el muestreo de espacios sin limitación en la cantidad de dimensiones
	    \end{itemize}
	\end{frame}

  \section{Pendientes y Trabajos Futuros}
	\begin{frame}
	\frametitle{¿C\'omo Seguimos?}
	  \begin{block}{Temas Pendientes}
	    \begin{itemize}
		\item 
	    \end{itemize}
	  \end{block} \pause

	  \begin{block}{Trabajo Futuro}
	   \begin{itemize}
		\item Bubbles en habla (e.g., detección de rasgos para expresividad o emociones)
	    
	    \end{itemize}                     
	  \end{block}	    
	\end{frame}

\author[Christian, Miguel, Mail\'en]{Mail\'en G\'omez Mayol,\\Miguel Mart\'inez Soler,\\Christian Cossio Mercado}

\frame{\titlepage}

\appendix
\section{\appendixname}
    \subsection{Material Adicional}
	\begin{frame}
	  \frametitle{Gr\'aficos}

	\end{frame}
    \subsection{Tests de Significatividad}
	\begin{frame}
	  \frametitle{Burbujas vs. Complejidad}

	\end{frame}

	\begin{frame}
	  \frametitle{Tiempo de Respuesta vs. Complejidad}

	\end{frame}

\end{document}