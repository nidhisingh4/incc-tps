 \documentclass[runningheads,a4paper]{llncs}
\usepackage[utf8x]{inputenc}
\usepackage[pdftex]{graphicx}
%\usepackage{graphicx}
\usepackage{amssymb}
\usepackage{url}
\usepackage{a4wide}
\setcounter{tocdepth}{3}
\usepackage{multirow}
\usepackage[spanish]{babel}

\urldef{\mails}\path|{cgcossio,mailen.gomezmayol, miguelmsoler}@gmail.com|
\newcommand{\keywords}[1]{\par\addvspace\baselineskip
\noindent\keywordname\enspace\ignorespaces#1}

\begin{document}
\mainmatter

\title{Detección de Rasgos en la Identificación de Letras Utilizando Bubbles}
\titlerunning{Detección de Rasgos Utilizando Bubbles}
\author{Christian Cossio Mercado \and Mail\'en G\'omez Mayol \and Miguel Mart\'inez Soler}
\authorrunning{Detección de Rasgos Utilizando Bubbles}

\institute{Facultad de Ciencias Exactas y Naturales, UBA\\Buenos Aires, Argentina\\\mails }
\toctitle{Detección de Rasgos Utilizando Bubbles}
\tocauthor{Cossio Mercado, Gomez Mayol, Martinez Soler}

\maketitle
\begin{abstract}
El reconocimiento de letras\ldots
\keywords{Identificaci\'on de Letras, Detecci\'on de rasgos, Bubbles}
\end{abstract}

% + Revisión bibliográfica, 
% + experimento viejo
% + variación propuesta. 
% + Diseño experimentaldetallado 

\section{Introducci\'on}
\label{sec:Introduccion}
\ldots

\subsection{Objetivo}
Reconocer rasgos utilizados por una persona para reconocer letras, presentadas utilizando diferentes tipografías.

\subsection{Revisi\'on Bibliográfica}
\label{sec:RevisionBibliografica}
\ldots


\section{Hip\'otesis}

A partir de la definición particular del experimento de este trabajo, así como por lo registrado en trabajos anteriores relacionados con el reconocimiento de letras, se definió el siguiente conjunto de hipótesis a verificar:
\begin{enumerate}
 \item El conjunto de rasgos detectados no es idiosincrático al grupo de individuos participantes en el experimento (i.e., se trata de una forma de identificación del ser humano en general).
    \subitem Hipótesis nula (H0): El conjunto de rasgos detectados es propio de cada grupo de individuos analizado. 
    \subitem Refutación (Ref): Fuera del alcance de este experimento. Sólo se realizará el análisis de comparación de los rasgos obtenidos en \cite{FisetEtAl08:BubblesForLetters}.
 \item El uso de tipografías ampliamente conocidas (por ej., la de Coca-Cola) facilita el reconocimiento de las letras.
    \subitem Hipótesis nula (H0): El uso de tipografías conocidas no mejora la performance en comparación con otras letras de la misma complejidad.
    \subitem Refutación (Ref): En comparación con las otras dos tipografías, se necesitará menos información para letras famosas versus otras de similar complejidad.
 \item Aún en casos en los que el sujeto no cree haber visto --- o ni siquiera conoce --- alguna tipografía famosa, el porcentaje de aciertos mejora significativamente.
    \subitem H0: No hay variaciones en la cantidad de información requerida para aquellas tipografías reconocidas durante el experimento ni para aquellas definidas como conocidas.
    \subitem Ref: La mejora en la performance decrece desde las tipografías conocidas vistas en el experimento, las 
	    famosas conocidas pero no vistas, las tipografías famosas no conocidas y las tipografías no famosas de similar complejidad.
 \item A mayor complejidad de las letras, menor eficiencia en el reconocimiento \cite{PelliEtAl06:LetterIdentification}
    \subitem H0: No habrá variación en la eficiencia entre aquellas letras de mayor y menor complejidad.
    \subitem Ref: Existe una variación significativa de la eficiencia entre letras de alta y baja complejidad.
 \item Los rasgos de una letra (e.g., 'b') varían de acuerdo a la tipografía que se esté utilizando \cite{PelliEtAl06:LetterIdentification}.
    \subitem H0: No hay variaciones en los rasgos detectados para un misma letra, inclusive con diferentes tipografías.
    \subitem Ref: Las zonas relevantes (rasgos) de una letra se ven afectadas por las características de la tipografía en cuestión.
 \item Habrá una variación en los rasgos de la letra `n' con relación a la incorporación de la letra `ñ' en todas las tipografías (cf. \cite{FisetEtAl08:BubblesForLetters}).
    \subitem H0: No habrá cambios en los rasgos de la letra `n' por la incorporación de la `ñ'
    \subitem Ref: La incorporación de la `ñ' agrega rasgos a la n para permitir su identificación.
 \item Un observador ideal utilizará rasgos distintos a los que utiliza un ser humano para la identificación de las letras \cite{PelliEtAl06:LetterIdentification}.
    \subitem H0: Los rasgos utilizados por un observador ideal son los mismo que los que resultan del análisis de un ser humano.
    \subitem Ref: Hay variaciones significativas entre los rasgos utilizados por un observador ideal y los sujetos del experimento.
\end{enumerate}


\section{Dise\~no del Experimento}
\label{sec:DisenoExperimento}

\subsection{Participantes}

Del experimento participaron 6 personas, de entre 20 y 33 años, con visión normal o corregida.

\subsection{Est\'imulos presentados'}
Letras del alfabeto español, en mayúscula y minúscula (incluyendo ñ).
Tres tipografías: Arial, Kunstler y otra de letras reconocidas (denominadas Famosas).

Se generó 16 bloques de 100 estímulos cada una, donde se presentó mitad de los estímulos en mayúsculas y luego la otra mitad en minúscula.

A cada estímulo se le aplicó un filtrados por bandas de frecuencia, de manera de mostrar información de 32 a 16 ciclos/letra,

\begin{figure}
 \includegraphics[scale=0.60]{graficos/letras.png}
  \caption{Conjunto completo de letras utilizadas en el experimento}
  \label{figura:conjuntoLetras}
\end{figure}


\subsection{Procedimiento}
Se presentan letras en forma individual por 200ms, para que luego el sujeto presione una tecla para indicar cuál fue la letra que se le acaba de mostrar.


\section{Resultados}
\ldots

\begin{table}
\centering
\label{tabla:cantidadBurbujas}
\caption{Cantidad de burbujas promedio necesarias para la identificación de las letras}
\begin{tabular}{c|r|r|r|c|r|r|r}

\hline
\multirow{2}{*}{\textbf{Letra}} & \multicolumn{3}{|c|}{\textbf{Tipografía}} & \multirow{2}{*}{\textbf{Letra}} & \multicolumn{3}{|c}{\textbf{Tipografía}} \\
\cline{2-4}\cline{6-8}
    & \textbf{Arial} & \textbf{Kunstler} & \textbf{Famosas} &     & \textbf{Arial} & \textbf{Kunstler} & \textbf{Famosas} \\\hline\hline
\textbf{A}   & 14   &   35   &   16   &   \textbf{a}   & 14   &   28   &   20   \\\hline
\textbf{B}   & 14   &   35   &   22   &   \textbf{b}   & 18   &   37   &   24   \\\hline
\textbf{C}   & 12   &   26   &   26   &   \textbf{c}   & 14   &   30   &   22   \\\hline
\textbf{D}   & 14   &   31   &   22   &   \textbf{d}   & 18   &   31   &   16   \\\hline
\textbf{E}   & 16   &   31   &   20   &   \textbf{e}   & 20   &   37   &   18   \\\hline
\textbf{F}   & 22   &   31   &   18   &   \textbf{f}   & 20   &   37   &   22   \\\hline
\textbf{G}   & 16   &   26   &   30   &   \textbf{g}   & 16   &   28   &   18   \\\hline
\textbf{H}   & 20   &   31   &   22   &   \textbf{h}   & 18   &   30   &   18   \\\hline
\textbf{I}   & 18   &   39   &   24   &   \textbf{i}   & 20   &   30   &   24   \\\hline
\textbf{J}   & 16   &   41   &   24   &   \textbf{j}   & 22   &   26   &   24   \\\hline
\textbf{K}   & 12   &   30   &   16   &   \textbf{k}   & 14   &   33   &   22   \\\hline
\textbf{L}   & 16   &   31   &   26   &   \textbf{l}   & 26   &   30   &   20   \\\hline
\textbf{M}   & 14   &   39   &   14   &   \textbf{m}   & 16   &   28   &   16   \\\hline
\textbf{N}   & 16   &   39   &   18   &   \textbf{n}   & 20   &   26   &   22   \\\hline
\textbf{Ñ}   & 18   &   39   &   12   &   \textbf{ñ}   & 20   &   26   &   22   \\\hline
\textbf{O}   & 16   &   30   &   22   &   \textbf{o}   & 16   &   33   &   22   \\\hline
\textbf{P}   & 14   &   33   &   22   &   \textbf{p}   & 16   &   30   &   22   \\\hline
\textbf{Q}   & 18   &   35   &   18   &   \textbf{q}   & 24   &   37   &   20   \\\hline
\textbf{R}   & 12   &   28   &   14   &   \textbf{r}   & 26   &   33   &   26   \\\hline
\textbf{S}   & 12   &   35   &   16   &   \textbf{s}   & 14   &   28   &   22   \\\hline
\textbf{T}   & 18   &   33   &   20   &   \textbf{t}   & 20   &   28   &   18   \\\hline
\textbf{U}   & 16   &   26   &   28   &   \textbf{u}   & 18   &   26   &   16   \\\hline
\textbf{V}   & 12   &   33   &   30   &   \textbf{v}   & 20   &   37   &   22   \\\hline
\textbf{W}   & 16   &   37   &   18   &   \textbf{w}   & 14   &   33   &   22   \\\hline
\textbf{X}   & 10   &   31   &   14   &   \textbf{x}   & 10   &   26   &   10   \\\hline
\textbf{Y}   & 14   &   28   &   14   &   \textbf{y}   & 16   &   24   &   14   \\\hline
\textbf{Z}   & 12   &   28   &   18   &   \textbf{z}   & 12   &   28   &   12   \\\hline
\end{tabular}
\end{table}

\subsection{Mediciones realizadas}
\ldots
\begin{itemize}
 \item Porcentaje de aciertos promedio por estímulo
 \item Porcentaje de aciertos y tiempo promedio de respuesta en función de la complejidad
 \item 

\end{itemize}


\section{Discusi\'on}
\ldots

\section{Conclusiones}
\ldots


\newpage

\begin{thebibliography}{20}
  \bibitem{AttneaveArnoult:QuantitativeStudyOfShape}	Attneave, F. \& Arnoult, M.: The quantitative study of shape and pattern perception (1956)
  \bibitem{Chauvin05:TestsForClassificationImages}	Chauvin, A., Worsley, K.J., Schyns, P., Arguin, M. \& Gosselin, F.: Accurate statistical tests for smooth classification images (2005)
  \bibitem{FarellPelli99:MeasureThreshold}		Farell, B. \& Pelli, D.: Psychophysical methods, or how to measure a threshold and why (1999)
  \bibitem{FisetEtAl08:BubblesForLetters}		Fiset, D. \& Blais, C., \'Ethier-Majcher, C., Arguin, M., Bub, D., Gosselin, F.: Features for Identification of Uppercase and Lowercase Letters (2008)
  \bibitem{FisetEtAl09:SpatioTemporalBubbles}		Fiset, D. et al: The spatio-temporal dynamics of visual letter recognition (2009)
  \bibitem{GosselinSchyns01:Bubbles}			Gosselin, F. \& Schyns, P.: Bubbles. a technique to reveal the use of information in recognition tasks (2001)
  \bibitem{GosselinSchyns03:BubblesUsersGuide}		Gosselin, F. \& Schyns, P.: A User's Guide to Bubbles (2003)
  \bibitem{GraingeEtAl08:LetterPerception}		Grainger et al: Letter Perception. from pixels to pandemonium (2008)
  \bibitem{OrucLandy09:ChannelSwitching}		Oruc, I. \& Landy, M.: Scale dependence and channel switching in letter identification (2009)
  \bibitem{PelliFarell10:PychophysicalMethods}		Pelli, D. \& Farell, B.: Psychophysical Methods (2010)
  \bibitem{Pelli01:HowWeSeeLetters}			Pelli, D.: How We See Letters. Implications for Making Better Displays (2001)
  \bibitem{Pelli03:InnefOfWordRecognition}		Pelli, D.: The remarkable inefficiency of word recognition (2003)
  \bibitem{PelliEtAl09:GestaltInLetterIdentif}		Pelli, D. et al: Grouping in Object Recognition. The role of a Gestalt law in letter identification (2009)
  \bibitem{PelliEtAl06:LetterIdentification}		Pelli, D., Burns, C.W., Farell, B. \& Moore-Page, D.C.: Feature detection and letter identification (2006)
  \bibitem{PetitGrainger02:PrimingForLetterPercept}	Petit, J.P. \& Grainger. J.: Masked partial priming of letter perpection (2002)
  \bibitem{PetitEtAl06:MaskedPrimingERP}		Petit, J.P. et al: On the time course of letter perception. A masked priming ERP investigation (2006)
  \bibitem{SolomonPelli94:VisualFilterForLetterIdent}	Solomon, J. \& Pelli, D.: The Visual Filter Mediating Letter Identification (1994)


\end{thebibliography}

\end{document}
